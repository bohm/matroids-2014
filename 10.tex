\input mbheader
\def\Bas{\mathcal{B}}
\title{10. cvičení z Matroidů}{grafove matroidy}

\ex{Opakovani: dokazte, ze $K_{3,3}^*, K_{5}^*$ nejsou grafove matroidy.}

\ex{Mejme orientovany $3$-souvisly graf $\vec{G}$. Zodpovezte na nasledujici otazky:

\begin{enumerate}
\item Necht $\vec{G}$ je take rovinny. Co je orientovana kruznice, to je jasne. Jak byste definovali orientaci pro kokruznice (v reci primalu)?

\item Necht $\vec{G}$ neni rovinny. Jak definovat orientovanou kokruznici v takovemto grafu?

\item Necht pro $\vec{G}$ plati, ze neobsahuje zadnou orientovanou kokruznici. Jak tuto vlastnost
popiseme v reci teorie grafu?
\end{enumerate}
}

\ex{V dukaze na minule prednasce se pouzilo to, ze mame-li binarni
matroid reprezentovany jako $[I_r | D]$, a nejakou kruznici-nadrovinu,
ktera neni fundamentalni kruznici vuci $I_r$, tak po relaxaci teto
kruznice-nadroviny by reprezentace \uv{vypadala jako} $[I_r |D]$, ale
to nemuze, takze vysledny matroid nemuze byt binarni.

Otazka zni: muze nejaky binarni matroid po relaxaci kruznice-nadroviny
zustat binarni? Popiste vsechny situace, kdy to je mozne.
}

\ex{Necht $G$ a $H$ jsou (vrcholove) $3$-souvisle multigrafy bez
izolovanych vrcholu a smycek. Dokazte (pokud to nebylo na prednasce),
ze pokud $M(G) \cong M(H)$, tak $G ≅ H$ (Pro $2$-souvisle je to s
upravou Whitneyho veta.)

\textbf{Napoveda:} Kazdy graf je jednoznacne urceny tim, ze zadate kazdemu vrcholu seznam jeho incidentnich hran.
}

\medskip\hrule\medskip

Uz jsme mluvili o tom, ze pro graf $G$ s $m$ hranami jeho prostor
cyklu a prostor rezu neni nic jineho, nez dualita kruznic a
kokruznic. V teorii grafu se to ale standardne definuje jinak:

Prostor cyklu $V_c$ je vektorovy prostor v $Z_2^m$ takovy, ze obsahuje
kazdy \textit{sudy} podgraf, cili podgraf se vsemi sudymi stupni. Da
se snadno nahlednout, ze baze tohoto prostoru je mnozina kruznic.
Prostor rezu je pak vektorovy prostor ortogonalni na $V_c$, rovnez nad
$Z_2^m$.

Pojdme si tedy dokazat, ze to v matroidech opravdu sedi:

\ex{Necht $A = [I_r | D] $ je binarni matroid (resp. nejaka jeho reprezentace) ranku
$r$ s nosnou mnozinou $m$. Zadefinujme \textit{prostor kruznic}
$V_\Cyc$ matroidu $A$ jako vektorovy prostor nad $Z_2^m$ takovy, ze
jeho baze je mnozina kruznic $A$. Dokazte, ze prostor kokruznic $A$ je
prostor ortogonalni k $V_\Cyc$ a je ranku $r$.
}

\input mbfooter
