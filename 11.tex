\input mbheader
\def\Bas{\mathcal{B}}
\title{11. cvičení z Matroidů}{Radovy matroidy jako zobecnění Hallovy věty}

Celym cvicenim bude prostupovat bipartitni graf na mnozinach $(X,Y)$.
Podle struktury na $Y$ budeme definovat matroid na mnozine $X$. Pro
podmnoziny $K ⊆ X$ budeme casto chtit najit vsechny sousedy z $Y$;
takovou mnozinu sousedu pak budeme znacit jako $N(K)$.

\ex{Nejprve zadefinujme matroid na $X$ tak, ze kruznice budou vsechny do inkluze
minimalni podmnoziny $K ⊆ X$ takove, ze $|N(K)| < |K|$. Dokazte, ze takova definice
skutecne urcuje matroid.

Pokud definice je spravna, tak se tento matroid jmenuje \textit{transverzalni matroid}.
}

\ex{Transverzalni matroid jiz mame hotovy, nicmene matroid, ktery nas
skutecne bude zajimat, se jmenuje \textit{Raduv matroid}.

Raduv matroid se definuje rovnez na mnozine $X$ pomoci bipartitniho
grafu $(X,Y)$, avsak mnozina $Y$ uz je nejaky predchozi matroid.
V Radove matroidu je kruznice $K⊆ X$ takova do inkluze minimalni mnozina,
pro kterou plati, ze $r(N(K)) < |K|$.

Overte, ze i Radova definice skutecne urcuje matroid.

\textbf{Napoveda:} Muze se hodit spocitat $r(N(C_1 ∪ C_2))$.
}

\ex{
\begin{itemize}
\item Ukazte, ze transverzalni matroid je take Raduv matroid.
\item Ukazte, ze kazdy matroid muze vzniknout jako Raduv matroid.
\end{itemize}
}

\ex{Uz vime, jak vypadaji kruznice v Radove matroidu; jak ale vypadaji nezavisle mnoziny?
Zde prichazi na scenu ono zobecneni Hallovy vety, tak jej dokazme:

\textit{Mnozina $K$ v Radove matroidu $M(X)$ je nezavisla, prave kdyz existuje mnozina $L ⊆ Y$ takova,
ze $L$ je nezavisla v $M(Y)$ a existuje perfektni parovani mezi $K$ a $L$.}
}

\ex{K cemu jsou Radovy matroidy uzitecne? Pokud si vzpominate, tak bipartitni argumenty
se pouzivaji v dukazu Matroid Intersection Theorem; neni tedy zadny div, ze se daji pouzit
k lepsimu nahledu na MIT (pokud je clovek zna).

Dokazme si tedy pribuzny vysledek Matroid Union pomoci Radovych matroidu:

\textit{Bud $M_1$, $M_2$ matroidy na dvou nosnych mnozinach, mozna
sebe pronikajicich.  Vlozme $I$ do systemu $\Ind$, pokud existuji
disjunktni mnoziny $I_1,I_2$ takove, ze $I = I_1 ∪ I_2$ a $I_i$ je
nezavisla v $M_i$. Pak $(E(M_1) ∪ E(M_2), \Ind)$ popisuje matroid.
}

}

\medskip\hrule\medskip
Na pozdejsi rozmysleni:

\begin{enumerate}

\item Pomoci Radovych matroidu dokazte silnejsi verzi Matroid Union,
tedy dokazte navic, ze pro kazdou mnozinu $X$ v matroidu $(E(M_1) ∪
E(M_2), \Ind)$ je jeji rank 

\[ (X) = \min_{F ⊆ X} |X ∖ F| + r_{M_1}(F) + r_{M_2}(F). \]

\item Zminovali jsme, ze pomoci Matroid Intersection jde dokazat
silnejsi verze Matroid Union (ta z predchoziho bodu). Rozmyslete,
jestli to jde i naopak, tj. z Matroid Union dokazat Matroid
Intersection Theorem.

\end{enumerate}

\input mbfooter
