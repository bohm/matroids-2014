\input mbheader

\title{1. cvičení z Matroidů}{co matroid je a co není}

\dfn[maticovy matroid]{Maticovy matroid vznikne z matice $m × n$ tak, ze kazdy sloupecek chapeme
jako jeden element z nosne mnoziny a mnozina elementu je nezavisla $⇔$ vektory jsou linearne nezavisle.
}

\dfn[grafovy matroid]{Grafovy matroid vznikne tak, ze vezmeme {\it hrany} grafu $G$ jako elementy nosne mnoziny
a mnozina elementu je nezavisla $⇔$ mnozina hran neobsahuje kruznici.
}

\dfn[kruznice]{ Kruznice $C ≡ $ jakakoli minimalni zavisla mnozina.}

\dfn[baze]{ Baze $B ≡$ jakakoli maximalni nezavisla mnozina.}

\dfn[rank]{ $r: 2^{E} → ℕ_0$,  $r(X) ≡ $ velikost nejvetsi nezavisle uvnitr $X$.}

\dfn[uzaver]{ $cl: 2^{E} → 2^{E}$, $cl(X) = \{y| r(X ∪ {y}) = r(X) \}$.}

\dfn[uzavrena mnozina n. flat]{ Kazda $X$ takova, ze $cl(X) = X$.}

\dfn[nadrovina]{ Nadrovina $H ≡$ maximalni mnozina, pro kterou plati $r(H) ≠ r(B)$.}

\medskip\hrule\medskip

\ex{Rozhodnete, jestli nasledujici struktury jsou matroid:

\begin{itemize} 
\item Mame graf $G$, elementy matroidu jsou hrany. Elementy jsou nezavisle, pokud spolu tvori parovani (ne nutne maximalni, proste nejake). Pocitame
i prazdne parovani.
\item Vezmeme Fanovu rovinu (konecnou projektivni rovinu na 7 vrcholech). Elementy matroidu budou body roviny, a mnozina bodu je nezavisla, pokud zadne tri body nelezi na primce.
\item Mejme cislo $k≥3$ a hypergraf $H$ (nosna mnozina $X$ a system podmnozin $H$). Elementy matroidu jsou hrany (mnoziny) z $H$. Elementy $E_1,E_2, …, E_k$ 
jsou nezavisle, pokud zadny vrchol z $⋃_i E_i ⊆ X$ neni pokryty $k$ mnozinami/elementy $E_i$.
%TODO: jeste jeden?

\end{itemize}

}

\sol{
\begin{itemize}
\item Prvni struktura neni matroidem, protoze ne vzdy jde mensi parovani rozsirit na vetsi pomoci pouze pridavani hran.
\item Fanova rovina matroidem je, dokonce reprezentovatelnym.
\item Podobny argument jako v prvni -- muzeme mit $k-1$ kopii mnoziny $\{1,2,3\}$ jako nezavislou velikosti $k-1$, a pak $k/3$ kopii kazde z mnozin
$\{1,4\} , \{2,4\}, \{3,4\}$. Obe jsou nezavisle, ale doplnit jednu z druhe neumime.
\end{itemize}
}

\ex{
\begin{itemize}
\item Jak byste popsali reci grafu pojmy {\it uzaver, uzavrena mnozina (flat), nadrovina}\/? Ukazte na prikladech. 
\item Jak byste popsali reci matic ty same pojmy? Odpovida pojem {\it nadrovina} intuici z matic v $ℝ^d$?
\end{itemize}
}

\sol{
\begin{itemize}
\item {\it uzaver} $X$ -- mnozina hran, ktere tvori kruznice s kostrou $X$. Nebo mnozina vektoru, ktere jsou v podprostoru urcenym $X$.
\item {\it nadrovina} $X$ -- maximalni mnozina hran, ktere neobsahuji kostru $X$. Maximalni mnozina vektoru, ktera nema plnou dimenzi.
\item {\it kokruznice} $X$ -- minimalni mnozina hran, ktera protina kazdou kostru $X$. Minimalni mnozina vektoru, jejichz doplnek
nema plny rank.
\end{itemize}
}

\ex{ Mejme nasledujici matici:
\[
\begin{pmatrix}
1 & 0 & 0 & 1 & 1 & 0 \\
0 & 1 & 0 & 1 & 0 & 1 \\
0 & 0 & 1 & 0 & 1 & 1 \\
\end{pmatrix}
\]
Chapejme ji jednou jako matici nad $\Z_2$, a podruhe jako matici nad $\Z_3$. Jsou maticove matroidy techto dvou matic totozne?
}

\sol{Z Oxleyho. Nejsou.}

\ex{Jake vlastnosti se ztrati, kdyz prejdu od grafu ke grafovemu matroidu? Predstavme si, ze mame grafovy matroid $M(G)$ vznikly z jednoducheho grafu $G$, jenze tento matroid $M(G)$ je zadany orakulem, cili:

\begin{itemize}
\item zname celou mnozinu elementu $E$,
\item pro kazdou podmnozinu elementu $X⊆E$ se muzeme dotazat orakula, jestli plati $X ∈ \Ind$.
\end{itemize}

Jde rozhodnout (s libovolnou vypocetni silou) nasledujici otazky?

\begin{itemize}

\item Byl puvodni graf souvisly?
\item Obsahoval puvodni graf kliku na alespon $30$ vrcholech?
\item Obsahoval puvodni graf perfektni parovani?
\end{itemize}
}

\sol{

\begin{itemize}
\item Nejde rozhodnout, graf s dvema disjunktnimi trojuhelniky a graf s dvema trojuhelniky spojenymi vrcholem tvori
stejny matroid.

\item Jde rozhodnout. Je treba dokazat, ze klika na $30$ vrcholech
tvori jednoznacne urceny matroid, pak proste vyresime hledani
podmatroidu v $M(G)$. Pouzit muzeme tento popis: ``maximalni nezavisla
mnozina je velikosti $29$, a kazda dalsi hrana z ${30 \choose 2} - 29$
tvori kruznici o velikosti alespon $3$.''

Puvodni graf pak musi byt souvisly, jinak by vubec nemohl mit tolik
hran (pokud ignorujeme paralelni hrany, ale to muzeme). No a kdyz je
souvisly a ma tuto vlastnost, uz je to uplny graf na $30$ vrcholech.
\item Opet nelze rozhodnout, graf s vidlickou a graf s perfektnim parovanim nejde rozpoznat.

\end{itemize}

}

\ex{Popiste uniformni matroid $U_{m,n}$ pomoci matice nad vhodnym telesem -- nebo vymyslete algoritmus, jak tuto reprezentaci spocitat.}
\medskip
\goodbreak
\sol{ Popisu postup nad $ℝ$ -- zacnu s jednotkovou bazi $ℝ^m$. Kazda
$m-1$-tice vektoru mi urci nadrovinu. Nehlede na to, kolik techto
nadrovin je, vzdycky existuje vektor mimo tyto nadroviny. Tak ho
zvolim a vlozim jako dalsi vektor do reprezentace, a algoritmus opakuji.
}

\ex{
Mejme matroid $M$. Uz mame \emph{uzaver mnoziny}, tak si muzeme snadno
definovat uzavrene a otevrene mnoziny: mnozina je uzavrena, pokud
$cl(X) = X$, a mnozina je otevrena, pokud jeji doplnek je uzavreny.
Otazka zni: dostali jsme takto topologii? Dokazte nebo vyvratte.

Axiomy topologie:
\begin{enumerate}
\item $∅, E$ (cela nosna mnozina) jsou otevrene.
\item Sjednoceni libovolneho poctu otevrenych mnozin je otevrena mnozina.
\item Prunik konecne mnoha otevrenych mnozin je otevrena mnozina.
\end{enumerate}
}

\sol{
\begin{enumerate}
\item Jednoduche -- obe jsou uzavrene a zaroven otevrene.

\item Delam pruniky doplnku. Kdyby v pruniku doplnku chybel element
nezvedajici rank, tak chybi v nektere z uzavrenych mnozin -- spor.

\item Chtel bych ukazat, ze sjednoceni uzavrenych je uzavrena, ale to
neni obecne pravda -- pokud rozdelim Hamiltonovskou kruznici na 10
vrcholech na 1 hranu, jeden kus po 4 hranach $A$ a jeden kus po 5
hranach $B$, pak $A ∪ B$ neni uzavrena mnozina, ale $A$ a $B$ samotne
uzavrene jsou. Takze konecne matroidy topologii nakonec nejsou.

\end{enumerate}
}

\ex{Jde kazdy maticovy matroid popsat jako grafovy matroid nejakeho
grafu? Jde kazdy grafovy matroid popsat jako maticovy matroid pro
nejakou matici?}

\sol{Necekam, ze se stihne, ale nechavam pro chytre hlavy a jako motivace
pristich cviceni/prednasek. Jinak $U_{2,4}$ neni grafovy, a kazdy graf
se da reprezentovat matici incidence.
}

\input mbfooter