% The UCW Macro Collection: Verbatim environment
% Written by Martin Mares <mj@ucw.cz> in 2010 and placed into public domain
% -------------------------------------------------------------------------

% We'll use internal macros of plain TeX
\catcode`@=11

% New \frenchspacing, which doesn't leave unwanted spaces in text.
\def\frenchsp@cing{\sfcode`\.\@m \sfcode`\?\@m \sfcode`\!\@m%
\sfcode`\:\@m \sfcode`\;\@m \sfcode`\,\@m}

% Typesetting of one verbatim word: |word| (enable by \inlineverbon)

% Set if spaces should be rendered as "bath-tub" glyphs
\newif\ifshowspaces
\showspacesfalse

\def\make@ther#1{\catcode`#1=12\relax}
\def\setverb@tim{\tt\frenchsp@cing\catcode`\`=13\catcode`\'=13\n@ligs%
\let\do=\make@ther\dospecials\obeyspaces\showsp@ces}

{\catcode`'=13
\catcode``=13
\gdef\n@ligs{\def`{\relax\lq}\def'{\relax\rq}}}

{\obeyspaces
\global\chardef\sp@ce=32
\gdef\showsp@ces{\ifshowspaces\let =\sp@ce\fi}
\global\let =\ }

{\catcode124=\active % Make | active
\gdef\ttst@rt{\begingroup\setverb@tim\let|=\endgroup}
\gdef\inlineverbon{\catcode124=\active\let|=\ttst@rt}
\gdef\inlineverboff{\catcode124=12}
}

% Typesetting of verbatim blocks:
%	- the block is enclosed in balanced {}'s
%	- leading and trailing empty lines are ignored
%	- all other empty lines are converted to \verbemptylineskip's
%	- tabs are interpreted (one tab stop per \verbtabsize columns)
%	- \showspacestrue works here, too, but it does not apply to tabs

\newskip\verbstartskip		% vskip before \verbatim
\newskip\verbendskip		% vskip after \verbatim
\newskip\verbinterlineskip	% between adjacent non-empty lines
\newskip\verbemptylineskip	% instead of every empty line
\newskip\verbleftskip		% left and right margin
\newskip\verbrightskip

\verbstartskip=3pt plus 1pt minus 0.3pt
\verbendskip=\verbstartskip
\verbinterlineskip=0pt
\verbemptylineskip=5pt plus 2pt
\verbleftskip=0in
\verbrightskip=0pt plus 1fil

% Can be re-defined to customize the verbatim environment
\def\verblocaldefs{}

% Size of a single tab
\newcount\verbtabsize
\verbtabsize=8

\newdimen\v@rbspace
\newdimen\t@b
\newdimen\t@bwidth

% When a complete line is assembled in \box0, \verbship is called to add it
% to the main vertical list. You can override it to get e.g. line numbering.
\def\verbship{%
	\ifdim\wd0>0pt
		\ifdim\v@rbspace>0pt
			\penalty-100
			\vskip\v@rbspace
		\else
			\ifdim\v@rbspace=0pt
				\vskip\verbinterlineskip
			\fi
		\fi
		\line{\hskip\verbleftskip \box0 \hskip\verbrightskip}
		\v@rbspace=0pt
	\else
		\advance\v@rbspace by \verbemptylineskip
	\fi
}

\newcount\verbcnt
\def\v@rbend{\par\egroup\endgroup\vskip\verbendskip}
\def\v@rbl{\ifnum\verbcnt>0\{\fi\global\advance\verbcnt by 1\relax}
\def\v@rbr{\ifnum\verbcnt>1\}\else\v@rbend\fi\global\advance\verbcnt by -1\relax}

\def\v@rbparams{%
	\setverb@tim
	\raggedbottom
	\verbcnt=0
	\v@rbspace=-1000pt
	\catcode124=12
	\vskip\verbstartskip
	\let\par=\endb@x
	\obeylines
}

\def\startb@x{\setbox0=\hbox\bgroup}
\def\endb@x{\egroup\verbship\startb@x}

{\catcode`\^^I=\active
\gdef\setupt@bs{\catcode`\^^I=\active
\setbox0=\hbox{\ }\t@bwidth=\wd0\multiply\t@bwidth by \verbtabsize
\def^^I{\egroup\t@b=\wd0\divide\t@b by \t@bwidth%
\multiply\t@b by \t@bwidth%
\advance\t@b by \t@bwidth\advance\t@b by -\wd0\startb@x\box0\hbox to \t@b{}}%
}}

{
\catcode`[=1\catcode`]=2\catcode123=\active\catcode125=\active
\gdef\verbatim[\begingroup\v@rbparams\catcode123=\active\catcode125=\active%
\let{=\v@rbl\let}=\v@rbr\chardef\{=123\chardef\}=125\verblocaldefs\setupt@bs\startb@x%
]]

% Input file verbatim

\def\verbinput#1{\begingroup\v@rbparams\verblocaldefs\setupt@bs\startb@x\input #1 \egroup\endgroup}

% Let's hide all internal macros
\catcode`@=12
