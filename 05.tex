\input mbheader
\title{5. cvičení z Matroidů}{Matroid intersection}

\ex{Jako rozcvičku dokažte Königovu větu pomocí M.I.T. Königova věta je toto:

\textit{Velikost maximálního párováni v souvislém bipartitním grafu $G$ je rovna velikosti minimálního
vrcholového pokrytí v $G$.}

\textbf{Nápověda:} Začněte tím, že popíšete maximální párování jako průnik dvou matroidů.

} %4.1

\ex{Dokažte, že algoritmická verze \textsc{Matroid Intersection} pro 3
matroidy najednou je NP-těžký problém.

\textbf{Nápověda:} Dokažte to redukcí přes problém \textsc{Orientovaná
$s,t$-Hamiltonovská Cesta}, čili Ham. cesta v grafu $\vec{G}$ s
předepsanými vrcholy $s,t$ jako začátek a konec. Jeden matroid by mohl
popisovat neorientovanou verzi grafu $G$.
}

\sol{
\begin{itemize}
\item $M_1$: grafový matroid $G$.
\item $\Ind(M_2)$: Množiny hran $I ⊆ V$ takové, že $vst(s) = 0,  ∀v ∈ V-s: vst(v) ≤ 1$.
\item $\Ind(M_3)$: Množiny hran $I ⊆ V$ takové, že $vyst(t) = 0, ∀v ∈ V-t: vyst(v) ≤ 1$.
\end{itemize}

Je třeba ověřit, že $M_2, M_3$ jsou matroidy, ale není to těžké. Pokud existuje max.
nezávislá ve všech 3 najednou velikosti $n-1$, pak je to H. cesta z $s$ do $t$.

}

\ex{Dokažte větu o duhové kostře:

\textit{Mějme graf $G$ a nějaké libovolné hranové obarvení $c:E →
\{1,2,…,k\}$. Graf $G$ obsahuje duhovou kostru právě tehdy, když pro
každou podmnožinu $E' ⊆ E$, která obsahuje nejvýše $t-1$ barev, platí,
že $G ∖ E'$ má nanejvýš $t$ komponent.}

} %4.6

\ex{\textit{Herní intermezzo:} Mějme Maker/Breaker hru na grafu $G$:
Hráč $B$ začíná smazáním jedné hrany $e$ z $G$, a hráč $M$ poté vybere
jednu hranu $f ∈ G - e$, kterou zapečetí (už nepůjde smazat).  Dále se
hraje na $G - e - f$, dokud jsou hrany. Hráč $M$ vyhrává, pokud
množina zapečetěných hran v libovolnou chvíli tvoří kostru $G$, a hráč
$B$ vyhrává, pokud se tak nestane.

Dokažte, že hráč $M$ má vítěznou strategii právě tehdy, když $G$ obsahuje
dvě hranově-disjunktní kostry.

}

\sol{

Nechť graf $2$ hranově disjunktní kostry nemá; pak existuje rozdělení
$V_1 ∪ V_2 ∪ … ∪ V_k$ takové, že $E(V_1,V_2,…,V_k) ≤ 2(p-1) - 1$. Pak
má Breaker hrát tak, že každé kolo eliminuje hranu z
$E(V_1,V_2,…,V_k)$. Vzhledem k počtu hran platí, že hráč $M$ stihne
zafixovat méně než $p-1$ z nich, a tedy bude výsledný graf nesouvislý.

Na druhou stranu: nechť existují dvě disjunktní kostry. Budeme hrát
tak, že pokud hráč $B$ smaže červenou hranu, podíváme se na komponenty
červené kostry $C_1$ a $C_2$ a spojíme zelenou hranou tyto
komponenty. Po spojení zkontrahujeme vrcholy, které jsme spojili.
Pokud $B$ smaže smyčku, hrajeme libovolně tak, abychom nevytvořili kružnici.

Zbývá dokázat, že $M$ vyhraje. Pokud $B$ má šanci vyhrát, tak jen tak,
že vytvoří řez $R_1, R_2$. Podíváme se na poslední hranu z řezu,
kterou smazal.  Pokud je červená, tak existují komponenty $C_1$ a
$C_2$ tak, že $C_1 ⊆ R_1$ a $C_2 ⊆ R_2$. Ale to je spor, protože v
tuto chvíli zvolíme zelenou hranu tak, abychom mohli spojit
komponenty.

}



\ex{Z přednášky známe velkou větu \uv{Matroid intersection theorem},
takže vás možná napadlo, jestli existuje \uv{Matroid union
theorem}. Opravdu existuje a jeho formulace hovoří o matroidu, který
jsme viděli již v předchozích cvičeních, a to je nedisjunktni
sjednocení dvou matroidů.

Dokážte tedy následující větu:

\textit{Mějme $M_1 = (E_1,\Ind_1)$ a $M_2 = (E_1,\Ind_2)$ matroidy ($E_1 ∩ E_2$ může být neprázdný).
    Vytvořme $M = (E_1 ∪ E_2, \{I_1 ∪ I_2 | I_1 \in \Ind_1 , I_2 \in \Ind_2 \}$. Pak $M$ je matroid a jeho ranková funkce pro $U ⊆ E_1 ∪ E_2$ lze
spočítat jako:}

\[r(U) = \min_{T ⊆ U}\left(|U ∖ T| + r_1(T ∩ E_1) + r_2(T ∩ E_2) \right). \]


\textbf{Nápověda:}
\begin{enumerate}
\item Pokud zvolíme nějaké $U$, můžeme zvolit $M_a, M_b$ pro M.I.T. v závislosti na $U$ -- nemusíme je volit stejné
pro všechny volby $U$.
\item Jeden z matroidů, na které budeme chtít použít M.I.T., bude $M_1|U ⊕ M_2|U$.
\end{enumerate}

}

\sol{Máme v ruce $U$, budeme pro něj chtít zvolit matroidy $M_a$ a
$M_b$ takové, že $r(U) = max E'$ jako max. nezávislé v $M_a ∧
M_b$. $M_a$ bude podle nápovědy $M_1|U ⊕ M_2|U$.  $\Ind(M_b)$ bude
postaven na stejné nosné množině, čili $V = M_1|U ⊕ M_2|U$ (některé
elementy jsou v $U$ jednou, ale v $M_1 ⊕ M_2$ dvakrát). $\Ind(M_b)=
\{I ⊆ V | I $ neobsahuje žádný prvek a jeho duplikát zároveň $\}$.

Tvrdíme, že $r(U)$ je max. nezávislá v $M_a ∧ M_b$. Kdyby nebyla, tak
existuje větší nezávislá v $M_a ∧ M_b$ a tu bychom mohli přeložit jako
nezávislou v $M_1 ∪ M_2$.

No a když tomuto věříme, tak $r(U) = \min_{T ⊆ U} r_a(T) + r_b(U ∖ T)
= \min_{T ⊆ U} r_a(T) + |U ∖ T|$, protože $U ∖ T$ je nezávislou v
$M_b$ (sama neobsahuje žádné duplikáty).
}

\input mbfooter
