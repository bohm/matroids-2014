\input mbheader
\def\Bas{\mathcal{B}}
\title{7. cvičení z Matroidů}{Souvislost + sudoku}

\ex{Necht matroid $M$ je $k$-souvisly a ma alespon $2k-1$ elementu. Pak vsechny
kruznice a kokruznice obsahuji alespon $k$ elementu.}

\ex{Necht matroid $M$ je $k$-souvisly a ma alespon $2k-1$ elementu. Pak $M$
neobsahuje zadnou podmnozinu $X$ velikosti $k$ takovou, ze $X$ je kruznice a kokruznice zaroven.}

\ex{Muze existovat matroid se souvislosti rovnou nekonecno?}

\bigskip\hrule\medskip

Mejme klasickou mrizku ($9 \times 9$) sudoku, celou vyplnenou nejakymi cisly.
Jeden radek $r_i$, sloupec $s_i$ nebo blok $b_i$ \textit{zkontrolujeme}, pokud
projdeme cely utvar a ujistime se, ze v utvaru jsou prave cisla od $1$ do $9$,
kazde jednou.

Oznacme si $K$ mnozinu vsech radku, sloupcu a bloku (velikosti 27).
Mnozina $X \subseteq K$ uz \textit{zkontroluje cele sudoku}, pokud
si po zkontrolovani techto radku, sloupcu a bloku si uz muzeme byt jisti, ze
cele Sudoku ma vsude cisla podle pravidel.

Otazka samozrejme je, jaka je velikost nejmensi mnoziny utvaru, ktera zkontroluje
cele sudoku.

\ex{Zkuste najit (hranim si s problemem) nejakou malou mnozinu utvaru, ktera
zkontroluje cele sudoku. Potom zkuste pro tuto mnozinu overit, ze zadna jeji
podmnozina uz nekontroluje cele sudoku.}

\ex{Oznacme si $\Bas \subseteq 2^K$ mnozinu vsech do inkluze minimalnich mnozin
utvaru, ktera zkontroluje cele sudoku. Necht vam prozradim, ze $(K,\Bas)$ je
matroid. Jak to pomuze k reseni hlavni otazky?}

\ex{Pokud $(K,\Bas)$ je matroid, je to uniformni matroid?}

\ex{Dokazme si nejakou velmi zjednodusenou formu vymenneho axiomu: dokazte, ze
pokud $L \in \Bas$ a plati, ze prvni radek lezi v $L$ a druhy nelezi, tak
muzeme nahradit prvni radek za druhy a vysledna zmenena $L'$ splnuje $L' \in
\Bas$.}

\input mbfooter
