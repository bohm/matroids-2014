\input mbheader
\def\Bas{\mathcal{B}}
\title{9. cvičení z Matroidů}{Totalni unimodularita}

\ex{Nalezněte $A$ matici alespoň $3 × 3$, aby $A$ nebyla totálně
unimodulární. Může navíc $A$ obsahovat pouze prvky $-1$, $0$ a $1$? A
co když zakážeme i $-1$?}

\ex{Rozhodnete o pravdivosti nasledujicich vyroku:

\begin{itemize}
\item Matroid je regularni, prave kdyz existuje reprezentace $R$, ktera je totalne unimodularni.
\item Matroid je regularni, prave kdyz kazda reprezentace $R$ je totalne unimodularni.
\item Matroid je regularni, prave kdyz kazda reprezentace $R$ obsahujici jen $\{-1,0,1\}$ je totalne unimodularni.
% \item Je-li matroid grafovy, tak jeho kazda reprezentace $R$ obsahujici jen $\{-1,0,1\}$ je totalne unimodularni.
\end{itemize}
}

\ex{Odbocka z optimalizace: k cemu jsou totalne unimodularni matice uzitecne?

Mejme mnohosten (popisujici nejaky linearni program) zadany jako
soustavu $A \vec{x} ≤ \vec{b}$.  Dokazte, ze kdyz $A$ je totalne unimodularni,
tak kazdy vrchol mnohostenu je celociselny.

Muze se hodit Cramerovo pravidlo: mam-li soustavu $n$ rovnic o $n$
neznamych tvaru $S\vec{x} = \vec{t}$ kde $S$ ma nenulovy determinant,
tak mohu spocitat $i$-tou souradnici reseni jako $x_i =
\det(S_i)/\det(S)$, kde $S_i$ je matice $n × n$ vznikla nahrazenim
$i$-teho sloupce vektorem $\vec{t}$.

}

\ex{Mějme matici $A$ velikosti $m × n$, jejíž řádky jdou rozložit
na dvě skupiny $B$ a $C$. Nechť také platí:

\begin{itemize}
\item $A ∈ \{-1,0,1\}^{m × n}$,
\item každý sloupec obsahuje nejvýše 2 nenulové hodnoty,
\item Pokud mají dvě nenulové hodnoty v jednom sloupci $A$ stejné znaménko, tak jeden řádek
patří do $B$ a druhý do $C$.
\item Pokud mají dvě nenulové hodnoty v jednom sloupci $A$ různé znaménko, tak oba rádky
patří do $B$ nebo zároveň do $C$.
\end{itemize}

Dokažte, že $A$ je potom totálně unimodulární.
}

\ex{Dokažte, že $0/1$-matice incidence grafu je totálně unimodulární právě tehdy, když graf je bipartitní.}

\ex{Uvazme matici $R_{10}$, zapsanou nize:


\[ R_{10} = I_5 +  \begin{pmatrix}
1 & 1 & 0 & 0 & 1 \\
1 & 1 & 1 & 0 & 0 \\
0 & 1 & 1 & 1 & 0 \\
0 & 0 & 1 & 1 & 1 \\
1 & 0 & 0 & 1 & 1 \\
\end{pmatrix}
\]

Zodpovezte nasledujici otazky:

\begin{itemize}
\item Jaky je dual matroidu $R_{10}$?
\item Je matroid $R_{10}$ grafovy?
\item Je determinant kazde ctvercove submatice roven $\{-1,0,1\}$?
\item Da se prava $5 × 5$ cast matice $R_{10}$ doplnit minusky tak, aby uz predchozi bod platil?
\end{itemize}
}

\input mbfooter
