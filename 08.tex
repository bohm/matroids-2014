\input mbheader
\def\Bas{\mathcal{B}}
\title{8. cvičení z Matroidů}{Reprezentovatelnost}

\ex{Nechť je matroid $M$ reprezentovatelný nad $GF(q)$ a má rank
$r$. Pak $M$ má nejvýše ${q^r -1 \over q-1}$ elementů. Jak získáme
$GF(q)$-reprezentovatelný matroid, který má právě tolik elementů?}


\ex{Jak se převádí geometricky zapsaný matroid (například Vámosův
matroid) na standardní matroidový zápis, tj. $(E,\Ind)$?

Čemu by měla
odpovídat přímka v geometrické reprezentaci ve vektorovém prostoru
$V$, pokud matroid je nad tímto v. prostorem reprezentovatelný?
}

\bigskip\hrule\medskip
\ex{Dokažte, že pro každý binární matroid a $∀ C,C' ∈ \Cyc$ platí, že $C \triangle C'$ je disjunktní sjednocení
kružnic.}

\ex{Dokažte, že pro každý binární matroid a  $∀ C ∈ \Cyc, D ∈ \Cyc^*$ platí,
že $|C ∩ D|$ je sudý.}


\bigskip\hrule\medskip

\ex{Dokažte, že matroid $P_6$ je reprezentovatelný nad $GF(5)$. Matroid $P_6$ má geometrickou reprezentaci
na tabuli.}

\ex{Algebraické intermezzo: Co to je $GF(4)$? Jak si ho představit?
Jak se s ním počítá? Jaké jsou jeho tabulky násobení a sčítání?}

\ex{Dokažte, že matroid $P_6$ není reprezentovatelný nad $GF(4)$.
% ale jeho minory po odebrání/kontrakci jedné hrany už jsou. (Nemusíte ukazovat všechny, ukažte alespoň jeden.)
}

\input mbfooter
